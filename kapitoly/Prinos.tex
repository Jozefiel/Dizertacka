\chapter{Závery pre realizáciu v spoločenskej praxi a pre ďalší rozvoj vedy}

\pagestyle{fancy}
\fancyhf{}
\fancyfoot[CE,CO]{\thepage}

%\fancyfoot{\leftmark}
\lhead{Závery pre realizáciu v spoločenskej praxi a pre ďalší rozvoj vedy}


Najdôležitejšie prínosy a vízie do budúcnosti predloženej doktorandskej dizertačnej práce sú zhrnuté do troch nasledujúcich základných oblastí.

\textbf{Prínosy pre vedný odbor}

Prínosom pre vedný odbor Telekomunikácie je návrh algoritmov určených pre rýchle skenovanie objektov, segmentáciu časti tváre a   kinematických parametrov pohybu z obrazovej sekvencie, rozpoznávania obrazu a postupov počítačového videnia použiteľných nielen v danom odbore, ale aj v interdisciplinárnej oblasti akou je biomedicínske inžinierstvo. 

Výhodou navrhnutých postupov, algoritmov a programových nástrojov je ich transparentnosť, otvorenosť a možnosť ich ďalšej modifikácie pre konkrétne potreby a aplikácie. Uvedené riešenia korešpondujú s požiadavkami projektu CEKR. Práca sprístupňuje, približuje a demonštruje na konkrétnych aplikáciách možnosti použitia a výhody počítačového spracovania a vyhodnocovania parametrov pre potreby biomedicínskych, ale aj iných technických aplikácií, čím vytvára prienik telekomunikačnej problematiky do iných odvetví.  Navrhnuté postupy a algoritmy je možné tiež použiť napr. v strojníckom priemysle (vyhodnotenie excentrít, posunov, meranie kinematiky mechanických pohyblivých súčastí). 

Verifikácia výsledkov práce v medicínskej praxi postupne overuje ich správnosť a potvrdzuje aktuálnosť riešenej problematiky. Navrhnuté nástroje založené na moderných multimediálnych technológiách použité v technickej praxi obohacujú vedný odbor o nové princípy a postupy pri spracovaní a vyhodnocovaní, čím naplňujú hlavné ciele práce. Prínosom práce je aj aplikácia získaných poznatkov o problematike počítačového a strojového videnia do edukačnej aktivity univerzity. 

\textbf{Prínosy pre pra}x

Prínosy pre prax sú v reálnych možnostiach využitia výsledkov práce, o čom svedčí aj ich použitie pri diagnostike riasiniek respiračného epitelu na Ústavoch fyziológie  a patologickej fyziológie a Klinike detí a dorastu Jesseniovej lekárskej fakulty Univerzity Komenského v Martine. Nakoľko práca bola riešená s podporou vývojového prostredia LabVIEW 8.5, objavuje sa ďalšia výhoda. Je ňou tzv. Web Publishing, čo znamená, že navrhnuté algoritmy prepísané do podoby užívateľských aplikácií je možné spúšťať a obsluhovať cez internetové pripojenie. Táto skutočnosť umožňuje odborníkom v rámci republiky, prípadne z rôznych častí sveta v danom čase masovo spolupracovať na riešení určitého problému bez nutnosti cestovať.
Ďalšou výhodou je aplikovateľnosť práce do oblasti medicínskych informačných systémov, nakoľko výsledky analýzy je možné štandardizovať a istou formou distribuovať v informačnom systéme ako časť pacientovej anamnézy.

Praktická verifikácia výsledkov práce dokazuje, že riešená problematika spracovania obrazovej informácie v medicínskom prostredí je aktuálna a o danú problematiku počítačového videnia je veľký záujem v zmysle zvýšenia kvality diagnostiky, aj keď treba mať na zreteli, že počítačovo stanovená diagnóza má len podpornú úlohu v rozhodovaní odborníka.
Vízie do budúcnosti

Cieľom práce do budúcnosti je vytvorenie komplexného matematického modelu popisujúceho pohyb sledovaného objektu, ale aj jeho klasifikácia na základe popisných parametrov (príznakov). Toto má za úlohu vytvorenie sofistikovanej diagnostickej metódy podporujúcej rozhodnutia odborníka s dostatočnou kvalitou. Ďalším prínosom by mala byť automatizácia a urýchlenie diagnostiky. 
Vzhľadom na skutočnosť, že proces klasifikácie si vyžaduje pomerne veľké vstupné dáta na trénovanie a rozsah práce, samotná klasifikácia sa javí ako cieľ pre budúce smerovanie tejto práce. V náplni projektu CEKR sa počíta so zostavením dostatočne veľkej databázy záznamov a ich parametrov (pri použití vyšpecifikovanej snímacej zostavy), ktorá by bola schopná pokryť požiadavky na klasifikáciu. 

