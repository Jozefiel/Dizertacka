
\chapter{Medicínske pozadie výskumu} \label{kap:Motivacia}

\pagestyle{fancy}
\fancyhf{}
\fancyfoot[CE,CO]{\thepage}
%\renewcommand{\footrulewidth}{1pt}
\lhead{Medicínske pozadie výskumu}

Spánkové poruchy postihujú približne 30% svetovej populácie. Podľa súčasnej platnej
medzinárodnej klasifikácie spánkových porúch sú respiračné spánkové poruchy druhým
najčastejšie sa vyskytujúcim ochorením /cite{neviem}

Respiračné spánkové apnoické ochorenia delíme na:
obštrukčné (OSAS): dospelí pacienti, pediatrickí pacienti
centrálne (CSA): Cheine-Stokes dýchanie, Primárne, ...
zmiešané: kombinované CSA a OSAS

Syndróm obštrukčného spánkového apnoe (OSAS) je respiračná choroba, ktorá spôsobuje
prerušenie oronazálnej ventilácie pri pretrvávajúcom dychovom úsilí na aspoň 10 sekúnd a
opakuje sa viac ako 5 krát za hodinu (2). Na jej vzniku sa vo významnej miere podieľa znížená
priechodnosť horných dýchacích ciest a zmena mechanizmu mäkkých štruktúr horných dýchacích
ciest. Najvýznamnejším rizikovým faktorom vzniku OSAS je anatomické zúženie horných
dýchacích ciest spôsobené predovšetkým: obezitou, kraniofaciálnymi abnormalitami, hypopláziou
tváre, zväčšeným objemom mäkkých tkanív, hypertrofiou tonzíl, zväčšením uvuly a dĺžky
mäkkého podnebia (3).

K presnému stanoveniu OSAS diagnózy sa vykonáva špeciálne komplexné vyšetrenie nazývané
polysomnografia (PSG). Ide o celonočné monitorovanie pacienta pomocou snímačov a senzorov,
ktoré dávajú lekárovi informáciu o dýchaní, ronchopátii, činnosti srdca, charaktere spánku a jeho
jednotlivých cykloch, o polohe tela a o okysličovaní organizmu. PSG sa vykonáva v
akreditovanom spánkovom laboratóriu, ktoré musí byť vybavené potrebnou meracou technikou. Z
výsledkov sa následne určuje konkrétny spôsob liečby (4).

PSG je považovaná za štandard pre diagnostiku OSAS, avšak toto vyšetrenie je finančne
nákladné a časovo náročné. Navyše na Slovensku ale aj vo svete je problém s nedostatkom
spánkových laboratórií, čo spôsobuje dlhé čakacie doby.

Kvôli zníženiu potreby používania PSG adolescentnou skupinou pacientov bol vytvorený
validovaný dotazník nazývaný Klinický záznam o spánku (SCR). Ten pomocou kombinácie
informácií z klinického vyšetrenia, skúmania subjektívnych symptómov pacienta a skúmania
prítomnosti ADHD dokáže s určitou presnosťou identifikovať prítomnosť OSAS. Pozitívni
pacienti sú následne uprednostňovaní na vyšetrenie PSG (5).

Vyšetrenie pomocou kontaktných meracích prístrojov je pre pediatrických pacientov často krát
stresujúce. Deti sú nepokojné a nedokážu spolupracovať s doktorom, čo predlžuje vyšetrenie a
vnáša chybu merania do dotazníka. Z toho dôvodu je požiadavka na vytvorenie meracieho
systému, ktorý by dokázal zmerať požadované parametre bezkontaktne.

Alternatívnym riešením je vytvorenie skenovacieho zariadenia, ktoré dokáže s určitou
precíznosťou geometricky aj textúrovo reprodukovať pacienta v 3D formáte. Následné meranie
na existujúcom modeli je možné vykonávať bez prítomnosti pacienta, taktiež je možné ho
opakovať alebo dodatočne zmerať aj iné tvárové parametre. Pri softvérovom meraní tvárových
parametrov vzniká možnosť automatizácie.

\section{Aktuálny výskum v oblasti OSAS diagnostiky}

https://arxiv.org/pdf/1911.05628.pdf