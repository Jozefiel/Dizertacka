
\chapter*{Abstrakt} \label{kap:Abstrakt}

Predložená dizertačná práca prináša návrh multi-kamerového snímacieho systému na podporu diagnostiky vybraných respiračných ochorení. Vizuálny systém je založený na hĺbkových kamerách (RGB-D) s princípom merania doby letu (ToF). Práca prináša návrh typu, počtu a usporiadania obrazových senzorov, tiež návrh základných algoritmov pre kalibrovanie kamier a snímanie 3D scén. V práci je uvedená metodika na filtrovanie a opravu získaných mračien bodov (modelov) a tiež na detekciu kľúčových bodov na ľudskej hlave. Cieľom navrhovaného systému je vytvoriť nástroj na automatizovanú diagnostiku obštrukčného spánkového apnoe (OSA) v oblasti detského lekárstva. \newline


{\LARGE Abstract}

This dissertation thesis brings the development of multi-camera scanning system for supporting diagnostics of selected respiratory diseases. The visual system is based on depth cameras (RGB-D) working on time-of-flight (ToF) principle. The research answers the questions about the type, number or image sensors arrangement, also the basic algorithms for image sensors calibration and 3D scenes capturing. In the dissertation thesis, there is also presented the methods for filtering and restoration of point clouds (models) and detection of key points on the human head and face. The main objective of the designed system is to create a tool for automated diagnostics of obstructive sleep apnea (OSA) and apply it in the field of pediatric medicine.