
\pagenumbering{arabic} % Číslování stran arabskými číslicem
\setcounter{page}{1} % Počítání stran nastavit na 1
\chapter{Úvod} \label{kap:Uvod}

\pagestyle{fancy}
\fancyhf{}
\fancyfoot[CE,CO]{\thepage}
\renewcommand{\footrulewidth}{1pt}
\lhead{Úvod}





Syndróm obštrukčného spánkového apnoe je ochorenie, ktoré patrí medzi najčastejšie sa
%vyskytujúce spánkové problémy u rôznej vekovej skupiny ľudí. Komplexná diagnostika tejto
%poruchy prebieha v špecializovaných spánkových labolatóriách pomocou polysomnografie.
%Takéto vyšetrenie je však časovo a finančne nákladné, čo spôsobuje dlhé čakacie doby na
%vyšetrenie. Pomocným riešením je diagnostikovať rizikových pacientov, ktorí môžu byť následne
%uprednostnení na lekárske vyšetrenie. V praxi sa používa dotazník nazývaný „Klinický záznam
%o spánku“, ktorý subjektívnym spôsobom lekára určuje pravdepodobnosť ochorenia. Výsledky
%dotazníka sú však ovplyvnené subjektívnym hodnotením vyšetrujúceho.
%
%Objektivizácia vyšetrenia môže byť zvýšena pomocou skríningového nástroja, ktorý by dokázal
%nahradiť manuálne vykonávané merania parametrov tváre predurčujúce spomínané ochorenie.
%V dnešnej dobe sa špecifické vlastnosti merajú manuálne kontaktnými meracími prístrojmi. Tento
%spôsob merania je pomalý a stresujúci predovšetkým u padiatrických pacientov. Práve stres
%u pacientov častokrát znemožňuje objektívny prístup k potrebným informáciám. Bezkontaktnou
%metódou merania by sa zvýšila objektivita a presnosť merania. Taktiež by sa znížila časová
%náročnosť vyšetrenia a umožnilo by sa meranie parametrov, ktoré obvykle nie sú zaznamenávané.
%
%RGB-D kamery okrem farby dokážu zachytiť aj informáciu o hĺbke prostredia. Pomocou tejto
%informácie je následne možné reprodukovať geometrické vlasnosti tváre pacienta do 3D modelu.
%3D rekonštrukcia objektov je technický problém, ktorý sa uplatnuje v širokom spektre od herného
%priemyslu cez strojárstvo až k medicíne. Práve v medicíne je 3D modelovanie interdisciplinárne.
%Model pacienta môže byť využívaný pri rôznych vyšetreniach bez potreby jeho prítomnosti.
%S využívanám umelej inteligencie vzniká aj možnosť určitého stupňa automatizácie.
%
%O spomínaný skríningový nástroj prejavili záujem svetoví odbornící, ktorí sa špecializujú na vyšetrovanie spánkových porúch.



