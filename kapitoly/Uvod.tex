
\pagenumbering{arabic} % Číslování stran arabskými číslicem
\setcounter{page}{1} % Počítání stran nastavit na 1
\chapter{Úvod} \label{kap:Uvod}

\pagestyle{fancy}
\fancyhf{}
\fancyfoot[CE,CO]{\thepage}
\renewcommand{\footrulewidth}{1pt}
\lhead{Úvod}


Syndróm obštrukčného spánkového apnoe je ochorenie, ktoré patrí medzi najčastejšie sa
vyskytujúce spánkové problémy u rôznej vekovej skupiny ľudí \cite{lam2010obstructive}. Komplexná diagnostika tejto poruchy prebieha v špecializovaných spánkových laboratóriách pomocou polysomnografie. Takéto vyšetrenie je však časovo a finančne nákladné, čo spôsobuje dlhé čakacie doby na vyšetrenie \cite{lam2005craniofacial}.

Z tohto dôvodu sa v posledných rokoch výskum intenzívne zameriava na vývoj skríningového nástroja, ktorý by vedel klasifikovať pacientov do jednotlivých rizikových skupín. Síce ide o medicínu problematiku, na jej riešenie sa využívajú technológie spadajúce do oblasti telekomunikácii. Z aktuálnych výskumov vyplýva, že nové diagnostické metódy využívajú k predikcii 2D snímky a 3D modely pacientov.  
Pre získanie týchto informácií s využívajú moderné snímacie zariadenia, ktoré poskytujú kvalitné dáta. Cena týchto zariadení je pomerne vysoká. Taktiež momentálne neexistuje komplexný systém, ktorý by sa špecializoval výlučne na detekciu obštrukčného spánkového apnoe. Jednotlivé problémy sa riešia čiastkovo a nie sú plne automatizované.     

Výsledkom tejto práce je návrh cenovo dostupného snímacieho systému a vývoj príslušného algoritmu. Zariadenie pomocou RGB-D snímačov zachytáva priestorovú informáciu pacienta a umožňuje automatizovane meranie špecifické kranio-faciálnych parametre tváre. Taktiež poskytuje možnosť vytvárania dátového setu, ktorý bude nápomocný pri vývoji klasifikačných nástrojov v neskorších fázach výskumu. 
\newpage
V praktickej časti práce sme navrhli metodiku snímania a filtrovania dát. Opísali sme spôsob kalibrácie multi-kamerového systému, pri ktorom sme zisťovali vnútorné a vonkajšie parametre používaných kamier. Opísali sme jav multi-kamerovej interferencie vznikajúcej pri paralelnej spolupráci ToF kamier. Navrhli sme filtračnú metódu IMBM, ktorá potláča vplyv tejto interferencie na hĺbkových mapách. Navrhli sme algoritmus umožňujúci spracovanie prijatých obrazových informácii Kinect v2 v reálnom čase. Implementovali sme detektor kľúčových bodov tváre na identifikáciu pohybu objektu a pre automatizovanie merania euklidovských vzdialenosti kranio-faciálnych parametrov. Taktiež sme navrhli metodiku segmentácie tváre pomocou Mask R-CNN neurónovej siete. Tá slúži k rozdeleniu konzistentného mračna bodov vytvoreného z viacerých kamier do klasifikačných skupín (oko, ucho, tvár, ...). Takéto segmentovanie umožní vytvorenie jedinečnej databázy, ktorej dáta budu použité pri trénovaní OSA klasifikátora. Pre tento účel bol vytvorený samostatný snímací systém, ktorého cieľom je získavanie dát určených k trénovaniu segmentačnej siete. Ten je umiestnený v detskom spánkovom laboratórium na klinike detí a dorastu JLF UK v Martine. 


