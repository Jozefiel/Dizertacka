\chapter{Tézy dizertačnej práce}
\label{kap:tezy}
\pagestyle{fancy}
\fancyhf{}
\fancyfoot[CE,CO]{\thepage}
\renewcommand{\footrulewidth}{1pt}
\lhead{Tézy dizertačnej práce}



\begin{itemize}
	\item Návrh algoritmu umožňujúceho vytvorenie precízneho priestorového modelu objektu z viacerých kamier vrátane kalibrácie vizuálneho systému.
	\item Návrh algoritmu pre automatizovanú detekciu kľúčových bodov na objekte a automatizované meranie vybraných geometrických priestorových parametrov.
\end{itemize}



Riešenie úlohy definovanej v prvej fáze je založené na využití multi-kamerového systému, ktorý v reálnom čase dokáže zachytiť priestorovú informáciu snímanej scény. Jednotlivé priestorové informácie z kamier sa následne zlúčia do spoločného priestorového modelu, ktorý bude použitý pre riešenie druhej tézy.
Podstatnou problematikou je vzájomná kalibrácia multi-kamerového systému a návrh filtračného algoritmu, ktorým sa zvýši kvalita výstupných priestorových dát. 

Problematika druhej tézy môže byť riešená viacerými spôsobmi. Podstatným krokom je identifikácia kľúčových bodov alebo oblastí v priestorovom modely. Jedným z možných riešení je segmentácia RGB-D mapy, kde sa pomocou algoritmov klasifikujú regióny bodov. Z výsledku klasifikácie bude následne možné rekonštruovať iba vybrané regióny 3D priestorového modelu. Výsledkom bude množina mračien bodov, ktoré môžu vystupovať ako samostatné modely alebo budú reprezentovať ucelený konzistentný model.



