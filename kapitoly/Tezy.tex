\chapter{Ciele dizertačnej práce}
\label{kap:tezy}
\pagestyle{fancy}
\fancyhf{}
\fancyfoot[CE,CO]{\thepage}
\renewcommand{\footrulewidth}{1pt}
\lhead{Tézy dizertačnej práce}



\begin{itemize}
	\item Návrh algoritmu umožňujúceho vytvorenie precízneho priestorového modelu objektu z viacerých kamier vrátane kalibrácie vizuálneho systému.
	\item Návrh algoritmu pre automatizovanú detekciu kľúčových bodov na objekte a automatizované meranie vybraných geometrických priestorových parametrov.
\end{itemize}


Problematika riešená v rámci dizertačnej práce pochádza z lekárskeho klinického prostredia (Klinika detí a dorastu JLF UK, Martin). Návrhu témy predchádzala dlhoročná spolupráca medzi Žilinskou univerzitou a Jesseniovou lekárskou fakultou UK v Martine. Špecialisti z klinického prostredia aktívne vstupujú do jednotlivých fáz riešenia, čím poskytujú užitočnú spätnú väzbu v celom procese, stanovovanie limitov a referenčných hodnôt a nakoniec aj validáciu výsledkov. Splnené ciele predloženej dizertačnej práce budú implementované do komplexného nástroja automatizovanej diagnostiky vybraných ochorení. \newline

Riešenie úlohy definovanej v prvom cieli je založené na využití multi-kamerového systému, ktorý v reálnom čase dokáže zachytiť priestorovú informáciu snímanej scény. Znižovanie času snímania objektu predstavuje z lekárskeho i technického hľadiska veľký prínos (pri zachovaní definovanej kvality). Jednotlivé priestorové informácie z kamier sa následne zlúčia do spoločného priestorového modelu, ktorý bude použitý pri riešení druhého cieľa. Podstatnou problematikou je vzájomná kalibrácia multi-kamerového systému a návrh filtračného algoritmu, ktorým sa zvýši kvalita výstupných priestorových dát. 

\newpage
Náplň druhého cieľa môže byť riešená viacerými spôsobmi. Podstatným krokom je identifikácia kľúčových bodov alebo oblastí v priestorovom modeli. Jedným z možných postupov je segmentácia RGB-D mapy, kde sa pomocou algoritmov klasifikujú regióny bodov. Z výsledku klasifikácie bude následne možné rekonštruovať iba vybrané regióny 3D priestorového modelu. V týchto čiastkových modeloch možno potom vykonať spoľahlivejšiu detekciu kľúčových bodov a meranie vzdialeností, nakoľko nebudú do týchto algoritmov vstupovať irelevantné časti modelov. Výsledkom bude množina mračien bodov, ktoré môžu vystupovať ako samostatné modely alebo budú reprezentovať ucelený konzistentný model.


