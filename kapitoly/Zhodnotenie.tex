\chapter{Výsledky dizertácie s uvedením nových poznatkov}

\pagestyle{fancy}
\fancyhf{}
\fancyfoot[CE,CO]{\thepage}

%\fancyfoot{\leftmark}
\lhead{Výsledky dizertácie s uvedením nových poznatkov}

Dizertačná práca rieši návrh skenovacieho systému, ktorý bude použitý ako skríningový nastroj pri diagnostike spánkového obštrukčného apnoe. Navrhnuté algoritmy sú univerzálne použiteľné vo viacerých oblastiach, ale pozornosť je sústredená na skenovanie pediatrických pacientov. Dosiahnuté výsledky ukazujú možnosti využitia počítačového videnia a číslicového spracovania signálov v oblasti medicínskej problematiky, konkrétne využitia hĺbkových snímačov pre priestorovú rekonštrukciu pacientov. \newline  
 
Cieľom dizertácie bol návrh vizuálneho systému a algoritmu, ktorým sa zautomatizujú v súčasnosti manuálne vykonávané úkony vyšetrenia. Vytvorenie tohto systému môže viesť k rozvíjaniu a zlepšovaniu diagnostiky OSA pomocou obrazových informácií. Výhodou tohto systému oproti komerčným riešeniam je kombinácia cenovej dostupnosti a rýchlosti snímania. Daná skutočnosť má v budúcnosti viesť k vytvoreniu robustnej databázy OSA pacientov, ktorá bude použitá pri vytváraní klasifikačného systému v procese sofistikovanej diagnostiky. Kľúčovými nástrojmi boli algoritmy číslicového spracovania signálu a analýzy obrazu, čo predstavuje prienik telekomunikácií do oblasti medicíny. Algoritmy a postupy boli testované a odladené na referenčných modeloch a následne overené na reálnych dynamických objektoch. \newpage 

\noindent Podľa názoru autora je pôvodným vedeckým prínosom dizertačnej práce:
\begin{enumerate}
	\item Návrh multi-kamerového systému a algoritmov umožňujúcich rýchle vytvorenie 3D modelov objektov.
	\item Návrh filtračného algoritmu, ktorý potláča vplyv multi-kamerovej interferencie vznikajúcej pri paralelnej spolupráci ToF kamier. 
	\item Prvotný návrh algoritmu, ktorým sa vykonáva segmentovanie mračien bodov obsahujúcich tváre. Pre zlepšenie výkonnosti segmentácie bol vytvorený systém zabezpečujúci zber potrebných obrazových dát.	
\end{enumerate}


Výhodou navrhnutých postupov, algoritmov a programových nástrojov je ich transparentnosť, otvorenosť a možnosť ďalšej modifikácie pre konkrétne potreby a aplikácie.

