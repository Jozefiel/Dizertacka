\chapter{Záver}

\pagestyle{fancy}
\fancyhf{}
\fancyfoot[CE,CO]{\thepage}

%\fancyfoot{\leftmark}
\lhead{Záver}

Táto dizertačná práca bola zameraná na prácu s 2D obrazovými a 3D priestorovými dátami. Cieľom bolo navrhnúť snímací systém a algoritmus, ktorý by slúžil ako podpora pre medicínske aplikácie. V posledných rokoch sa medicínsky výskum intenzívnejšie zaoberá diagnostikou OSA pomocou 3D informácií pacientov. Síce tento problém je medicínsky, pre jeho riešenie a ďalší rozvoj je taktiež potrebný výskum v technickej oblasti. \newline

Navrhnutý algoritmus mal umožniť vytvorenie precízneho priestorového modelu objektu s použitím viacerých RGB-D kamier. Algoritmus mal taktiež umožniť detekciu vybraných kľúčových bodov na snímanom objekte a automatizovať meranie geometrických parametrov.  \newline

Systém bol optimalizovaný pre snímanie pediatrických pacientov, ktorí svojimi vlastnosťami špecifikovali určité kritériá a požiadavky. Základnou požiadavkou bolo minimalizovanie doby skenovania a bezkontaktné meranie euklidovskej vzdialenosti kranio-faciálnych parametrov.  \newline

V prvej časti sme navrhovali multi-kamerový systém a určovali sme pracovné podmienky. Experimentálne sme overili, že pre skenovanie detských pacientov je potrebné pracovať s viac-kamerovým systémom. Taktiež sme poukázali na nutnosť použitia paralelnej spolupráce kamier. Detailne sme opísali spôsob geometrickej a multi-kamerovej kalibrácie, ktoré su využívané v procese snímania.  \newpage

V druhej časti opisujeme algoritmus, ktorým je vykonávaný proces snímania. Našim návrhom sme vytvorili systém pre zber dát z viacerých RGB-D kamier v reálnom čase. Tým sme minimalizovali vplyv pohybu na výslednú presnosť 3D modelu. Navrhli sme systém, ktorý využíva detekciu kľúčových bodov na tvári k meraniu pohybu objektu počas doby skenovania. K zvýšeniu kvality zosnímaných dát sme implementovali známe filtračné a registračné metódy. K potlačeniu multi-kamerovej interferencie, ktorá je dôsledkom paralelnej spolupráce viacerých ToF kamier, sme navrhli vlastnú filtračnú metódu s názvom IMBM. Tá dokáže odstrániť interferované regióny v hĺbkovej mape a do určitej miery rekonštruovať tieto poškodené miesta. Pomocou 68 bodového detektora kľúčových bodov sme navrhli metódu merania euklidovských vzdialenosti vybraných kranio-faciálnych parametrov. Overili sme presnosť merania na živých objektoch a zároveň ako rozšírenie práce sme navrhli algoritmus, ktorý umožňuje segmentáciu ľudskej tváre. Takéto segmentovanie má potenciálne využitie pri procese prípravy datasetu, ktorý bude zložený z mračien bodov a bude použitý k vytvoreniu OSA klasifikátora. \newline 

Z dôsledku potrieb pri návrhu algoritmu bol taktiež vytvorený systém pre zber obrazových dát, ktorý je umiestnený v spánkovom laboratóriu. Tento systém zaznamenáva hĺbkové a RGB-D informácie pediatrických pacientov spolu s dokumentáciou o vyšetrení. 

V ďalšom vývoji a výskume sa zameriame na implementáciu novších cenovo dostupných hĺbkových snímačov. Cieľom bude zvýšiť presnosť snímania pediatrických pacientov. Taktiež sa budeme venovať zvyšovaniu presnosti segmentácie tváre. 
